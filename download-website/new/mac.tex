\documentclass{article}
\usepackage{amsmath}
\usepackage[mathletters]{ucs}
\usepackage[utf8x]{inputenc}
\setlength{\parindent}{0pt}
\setlength{\parskip}{6pt plus 2pt minus 1pt}
\usepackage[breaklinks=true]{hyperref}
\setcounter{secnumdepth}{0}

\title{Haskell Platform for OSX}
\author{}

\begin{document}
\maketitle

\subsection{Haskell Platform for Mac OSX}

For Mac OS X Leopard (10.5) and above:

\begin{itemize}
\item
  Download Haskell for Mac OS X
\end{itemize}
After downloading:

\begin{itemize}
\item
  Open the .dmg file
\item
  Follow the install instructions
\end{itemize}
\subsection{Mac Ports}

The Haskell Platform is also in
\href{http://macports.org}{MacPorts}. Once MacPorts is installed,
you can build the haskell platform by typing

\begin{verbatim}
     sudo port install haskell-platform
\end{verbatim}

\end{document}

